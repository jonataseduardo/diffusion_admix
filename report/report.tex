\documentclass[a4paper, 12pt]{article}

%\usepackage[portuges]{babel}
\usepackage[utf8]{inputenc}
\usepackage{amsmath,amssymb}
\usepackage{indentfirst}
\usepackage{graphicx}
\usepackage{multicol,lipsum}
\usepackage{libertine}
\usepackage{fancyhdr}
\usepackage{fullpage}
\usepackage{setspace}
\usepackage{natbib}
\usepackage[hidelinks]{hyperref}


\renewcommand{\baselinestretch}{1.4} 

\newcommand{\ts}{\textsuperscript}
\newcommand{\st}{\textsuperscript{\textit{st}}}
\newcommand{\bs}{$\blacksquare$}

\newcommand{\del}[1]{\ensuremath{\frac{\partial}{\partial #1}}}
\newcommand{\dell}[1]{\ensuremath{\frac{\partial^2}{\partial {#1}^2}}}
\newcommand{\dev}[1]{\ensuremath{\frac{d}{d #1}}}



\begin{document}
	\begin{center}
    \Large{Genetic Load and Efficacy of Selection on Admixed Populations} 
  \end{center}
	
	\begin{flushright}

   \begin{list}{}{
      \setlength{\leftmargin}{7.5cm}
      \setlength{\rightmargin}{0cm}
      \setlength{\labelwidth}{0pt}
      \setlength{\labelsep}{\leftmargin}}

      \begin{list}{}{
      \setlength{\leftmargin}{0cm}
      \setlength{\rightmargin}{0cm}
      \setlength{\labelwidth}{0pt}
      \setlength{\labelsep}{\leftmargin}}

  \item Candidate: \
    Jônatas Eduardo da Silva César\ts 1\\
      \item Supervisor: Diogo Meyer\ts 1\\
      \item \ts 1 {\small \textit Department of Genetics and Evolutionary Biology, Institute
          of Biosciences, University of São Paulo.}
      \item BEPE Supervisor: John Novembre\ts 2\
      \item \ts 2 {\small Department of Human Genetics, University of Chicago.}
      \end{list}
   \end{list}
\end{flushright}
\vspace{1cm}

\begin{abstract}
To understand the interplay between demographic and selective processes in
shaping genetic variation, two key quantities of interest are genetic load and
the efficacy of selection, which is the contribution of selection to the rate
of change of the genetic load. Demographic processes such as population splits,
bottlenecks and population growth can bring about changes in the magnitude of
genetic load and efficacy of selection. For humans, an emerging  consensus is
that among population differences in genetic load are likely to be small, while
differences in the efficacy of selection may be large, with African populations
having the highest estimated efficacy of selection.  Admixture processes, on
the other hand, can increase the genetic load and efficacy of selection above
naive expectations and are still not completely understood.

Given the increasing importance of admixture in shaping the genetic variation
of future human generations, we propose to theoretically and computationally
investigate the genetic load and efficacy of selection of admixed populations.
We aim to extend existing analytical treatments of load and efficacy of
selection to models involving arbitrary numbers of populations and rates of
admixture. We will also quantify the magnitude of load and efficacy of
selection in various admixed populations, including those present in the 1000
Genomes and a dataset of 1320 admixed Brazilian individuals of the AbraOM
project.  This study will be carried out in collaboration with Prof. John
Novembre who is a member of the Department of Human Genetics of the University
of Chicago and an expert in the fields of genetic load and admixture of human
populations.
\end{abstract}


\section{Introduction} 

Genetic load results from the accumulation of deleterious variants in
a population. According to the seminal work of \cite{Kimura1963a}, genetic load
at equilibrium is expected to be higher in populations with smaller effective
size.  The intuition behind this result is that as the intensity of drift
increases, there is a reduced efficacy in removing slightly deleterious
variants, which segregate in the population and thus contribute to polymorphism
(and thus to the load). In larger populations, where drift is a weaker force,
the deterministic selective processes are responsible for the fate of most
mutations, so deleterious variants are more efficiently purged from the
populations and as a consequence the resulting load is lower. 

Using this reasoning, \cite{Lohmueller2008} analyzed one of the first next
generation sequencing (NGS) datasets, and argued that the higher proportion of
non-synonymous over synonymous variants ($P_N/P_S$) in Europeans in comparison
with Africans was evidence of increased load in Europeans caused by a decrease
in the efficacy of selection during the Out-of-Africa bottleneck. 

More recently, \cite{Simons2014} and \cite{Do2015}, using large NGS datasets,
claimed that there are no significant differences in the average number of
deleterious alleles per individual when Europeans and Africans  are compared.
Both studies concluded that the population growth after the Out-of-Africa event
results in an increase in the efficacy of selection, canceling out the effect
of the population bottleneck in creating increased genetic load.

If such a cancellation due to demography occurred, why did
\citep{Lohmueller2008} identify an excess of load among non-Africans?
According to \cite{Koch2017a} as well as \cite{Simons2014} and \cite{Do2015},
the apparent contradiction between their results is due to the method used to
quantify load. The \cite{Lohmueller2008} study quantified load using $P_N/P_S$
(i.e. the ratio of number of segregating non-synonymous to synonymous
variants), which is a statistic that is only expected to document increased
load in equilibrium conditions. This occurs when populations have maintained a
constant population size for a long span of time, so that the opposing effects
of mutation (introducing deleterious variants), selection (removing them) and
drift (interfering with the efficacy of selection) reach a steady state. In
non-equilibrium conditions (e.g., when  populations  recently expanded) the
expectations of increase $P_N/P_S$ for populations with smaller effective
population sizes are no longer valid \citep{Koch2017a}.

\paragraph{Quantifying Genetic Load.} To better understand the current
knowledge on genetic load we will follow the discussion and terminology
introduced by \cite{Gravel2016}. Given alleles $a$ and $A$ in a locus $i$ the
Malthusian fitness values of the genotypes $aa$, $aA$ and $AA$ can be written
as 1, $1 + h_is_i$ and $1 + s_i$.  We assume $A$ to be the least favored allele
($s_i < 0$) and $0 \leq h_i \leq 1$ the dominance coefficient. If the allele
$A$ is present at a frequency $x_i$ in a random mating population  it will add
a value of $\delta \omega_i = s_i (2 h_i x_i + (1 - 2h_i)x_i^2)$ to the mean
fitness, relative to the fitness of the most favorable allele.

In the case of multiple independent  loci of small effect, the net fitness is
evaluated by $\omega = \prod_i (1 + \delta \omega_i) \approx 1 + \sum_i \delta
\omega_i$. Then, using the definition of the genetic load as the relative
fitness reduction compared to the optimal genotype, $L = (\omega_{max} -
\omega)/\omega_{max}$ and since by our definition $\omega_{max} = 1$ it follows
that, 
\begin{align} 
  L = - \sum_i s_i (2 h_i x_i + (1 - 2h_i)x_i^2).
  \label{eq:load} 
\end{align}

Notice from equation \eqref{eq:load} that in order to properly measure
differences in  genetic load  between populations we ideally need to know which
variants in the genome are deleterious so as to estimate their allelic
frequencies, fitness effects and dominance coefficients. Given that none of
these variables are known with great confidence, disagreement among studies are
often due disagreement regarding the assumptions on how to measure these
variables \citep{Henn2015a,Gravel2016}. One common strategy that was used by
\cite{Simons2014,Do2015} is to assume equal and constant additive fitness
effects ($h_i = 0.5$ , $s_i = s$) over all loci, making the estimate of genetic
load proportional to the average number of putatively deleterious alleles per
individual ($L_{add} = -s \sum_i x_i$).


In fact, the average number of deleterious variants per individual is a robust
summary statistic to test differences in load, since it always increases with
the population genetic load. Nevertheless, this statistic still depends on the
prediction of the deleterious phenotype for  alleles  and depending on the tool
which is chosen to predict deleterious effects, different conclusions can be
reached. For example, in contrast to the results of \cite{Simons2014},
\cite{Fu2014} used the software PhyloP \citep{Pollard2010} to impute the
non-synonymous deleterious variation (instead of Polyphen-2)
\citep{Adzhubei2010}, and in an analysis of 6500 exomes of African Americans
and European Americans found that there is a small but statistically
significant difference in the number of deleterious mutations  between these
populations.

Both studies backed their conclusions with simulations based on the European
Out-of-Africa demographic history, with the major difference being that
\cite{Fu2014} used selection parameters based on \cite{Eyre-walker2007}'s
distribution of fitness effects while \cite{Simons2014} explored  several
selection scenarios with constant fitness coefficients.  Moreover, as pointed
out by \cite{Henn2015b}, large differences in genetic load between human
populations can be measured  if estimates for  dominance ($h_i$) and fitness
coefficients ($s_i$) are used to estimate load according to \eqref{eq:load}.


\paragraph{Efficacy of Selection.} Despite the fact that large differences in
load between human populations are still in dispute, \cite{Gravel2016} showed
that the present day efficacy of selection significantly differs between modern
human populations. The efficacy of selection is defined as the rate at which
the genetic load changes over time and can be quantified using the allele
frequency of deleterious variants (more details in subsection \ref{sec:fit}).

Differences in efficacy of selection between populations have strong
implications for human evolution, implying that large differences in genetic
load may emerge  even if the present day differences are small.  Specifically,
\cite{Gravel2016} used the 1000 Genomes data and showed that the efficacy of
selection is significantly higher for African populations than for Asians.
Strikingly, in a result that was not theoretically explored by the author,
\cite{Gravel2016} showed that the efficacy of selection of North American
admixed populations was greater than expected with respect to the weighted
average of the efficacy of selection in the parental populations.  These
results suggest that non-linear phenomena result from the admixture processes
and are important to determine the genetic load of future human generations.


\section{Objectives}

% The literature focused on differences of
% load between the African and European populations
% \citep{Lohmueller2008,Simons2014,Fu2014}, and despite some studies which
% analyzed differences in load between several human populations
% \citep{Do2015,Henn2015b,Gravel2016}, most of the theoretical and computational
% investigations were supported by demographic simulations based on European and
% African demographic histories. 

Population admixture has had a significant role into shaping  human
genetic variability \citep{Wall2016}. Nevertheless, few studies using realistic
simulations investigated the fate of deleterious mutations within admixed
populations \citep{Harris2016a, Kim2017a}. For example, motivated by the
evidence provided by \citep{Sankararaman2014} that, in  Europeans,
the Neanderthal ancestry near conserved regions of the genome appears to be
depleted, \cite{Harris2016a} showed that similar patterns could be replicated
by realistic simulations of Human-Neanderthal split and admixture processes. In
a subsequent study, \citep{Kim2017a} explored other demographic parameters in
which such depletion of Neanderthal ancestry of deleterious alleles could be obtained.

Given the increasing importance of the admixture process into shaping the
genetic variation of future human generations we propose to investigate the
following question: 
\\\

\textit{How does the interplay between the relative contributions
  of source populations, dominance coefficients and the strength of selection
  determine the load and efficacy of selection of an admixed population ?}
\\\

To investigate this question  we will develop the theory of genetic load and
efficacy of selection for admixed populations using Kimura's diffusion
equations . This will be done by adapting the calculation done by
\cite{Gravel2016} (detailed in the subsection \ref{sec:fit} for the case of a
single population) with the inclusion of the appropriate terms that account for
admixture events,  similarly to those introduced by \cite{Jouganous2017}.
Furthermore, our theoretical predictions will be tested against simulated
datasets as well against whole-genome sequencing data of two admixed
populations.   

\section{Work plan}

In a first phase, we will carry out theoretical studies of the processes of
admixture using diffusion theory in order to dissect the contributions of the
effective population size, the proportion of source populations, the intensity
of selection, and the dominance coefficients to the genetic load and efficacy
of selection of the admixed population. In order to test our predictions, we
will perform a series of simple evolutionary simulations of admixture processes
with two source populations. By checking for consistency between the analytical
treatment and the simulations we will validate both of these approaches. 
% reler

In a second phase, we will test the predictions of the theoretical modeling of
genetic load and efficacy of selection against two genomic data sets of
admixture population. The first dataset is composed of the North American admixed
population of the 1000 Genomes project \cite{1000Genomes2015}. The analysis of
this data will be followed by extensive simulations based on the demographic
history inferred by \citep{Gravel2013}.  In this step we
will be able compare the estimation of load and efficacy of selection for the
real data with the expected values of simulations carried out using different
values of dominance and selection coefficients. 

The second data analysis will be done in collaboration with the \textit{"Centro
  de Estudos do Genoma Humano"}  from the University of São Paulo. Here, we
will study the genetic load and efficacy of selection of Brazilian admixed
population using a whole-exome sequencing data  of 1320 individuals of the
AbraOM project \citep{Naslavsky2017} which is funded under FAPESP process
\#2014/50931-3. We will restrict our analyses of this data by using only  the
subset of coding variants that passes the same quality control pipeline
proposed in \cite{Naslavsky2017} which consists in the use of GATK quality
control filters. 


\section{Materials and Methods}

\subsection{Simulations}

There are several tools in the literature that can be used to simulate
evolutionary processes with complex demography and to thoroughly explore the
parameter space and nuances of the theoretical predictions. We propose to
employ two layers of simulations. The first  will use the \textit{ms} software
\citep{Hudson2002} to carry out neutral coalescent simulations of the studied
populations, incorporating realistic demographic histories. The coalescent
neutral simulations are computationally efficient and will provide the null
distributions for the genetic variability since the load and efficacy of
selection statistics are equal to zero by definition.

For the second layer, we will use software such as \textit{moments}
\citep{Jouganous2017} and \textit{SLiM-2} \citep{Haller2016} to simulate the
site frequency spectrum  under several combinations of dominance and selections
coefficients. The software \textit{moments} implements the evolution in time of
the Kimura's diffusion equation for complex demographic scenarios including
admixture process with two or more populations. The software \textit{SLiM-2}
implements a  realistic simulations using the Wright-Fisher model that takes
into account the position of selected variants, recombination and dominance
parameters as well the distribution of fitness effects. 

With the proposed set of simulations we will be able to test the dependency of
load and efficacy of selection in admixed population with the proportion and
effective sizes of source populations, and to evaluate the effect of different
combinations of dominance and selection regimes. 

\subsection{Efficacy of selection derivation}
\label{sec:fit}

Differences in genetic load between populations can be understood in the light
of the concept of efficacy of selection, which is the capacity of natural
selection to increase the population fitness by removing deleterious variation.
This process counters the effects of drift and mutation which are forces that
can decrease the population fitness by increasing the frequency of deleterious
mutation and by introducing new deleterious mutations in the population,
respectively.

Nevertheless, only recently did \cite{Gravel2016} explore a precise
statistical test to measure differences in efficacy of selection based on the
temporal variation of genetic load and allele frequencies in large populations,
using the Kimura's diffusion equation \citep{Kimura1955,Kimura1955a}. 

Let the allele frequency moments be defined by $\mu_k \equiv \int_{0}^{1}x^k
\phi(x,t)dx$, the fitness in the diploid case can be written as 
\begin{equation}
  W = s[2h\mu_1 + (1 - 2h)\mu_2]
\end{equation} 
Then the rate of change in fitness will depend on the derivatives of the first
and second moment which can be calculate using the diffuion equation for
$\phi(x)$.  

The diffusion equation defines the temporal evolution of the site frequency
spectrum function $\phi(x,t)$, which gives the probability density that one
allele is at a frequency $x$ at a time $t$ in a randomly mating population of
size $N = N(t) \gg 1$. A modern version of the diffusion equation, as defined
by \cite{Ewens2004}, is given by,
\begin{align}
\label{eq:difeq}
\frac{\partial}{\partial t} \phi(x,t) & \approx 
\frac{1}{4N}\frac{\partial^2}{\partial x^2} 
x (1 - x) \phi(x,t) \\ \nonumber
& \quad -s\frac{\partial}{\partial x} 
(h + (1 - 2 h)x)
x (1 - x) \phi(x,t) \\ \nonumber
& \quad + 2 N u \delta(x - \frac{1}{2N}),
\end{align}
where $u$ is the mutation rate, $\delta$ is Dirac's delta function, and $s$
and $h$ are constant over time. From this equation we can calculate the
evolution for the allele frequency moments.

In the infinite allele model there is pontentially a infinity number of alleles
ate frequency 0 and 1 so it is convinient to write 
\[
\mu_k \equiv \int_{0}^{1}dx x^k \phi(x,t)dx 
= \int_{0^+}^{1^-}dx x^k \phi(x,t) + K_0 \delta_{0,k} + K_1
\]
where $K_0$ and $K_1$ are the number of sites at frequency 0 and 1 and 
$\delta_{0,k}$ is the Kronecker's delta. 

\begin{align}
  \dot \mu_k 
  &= \int_{0^+}^{1^-}dx x^k \del t \phi(x,t) 
  + \dot K_0 \delta_{0,k} + \dot K_1 
  \nonumber \\
  &= \del t \int_{0^+}^{1^-}dx x^k \phi(x,t) 
  + \dot K_0 \delta_{0,k} + \dot K_1 
\end{align}

Writing the diffusion equation as 
\[
  \del t \phi (x) =  
  \frac{1}{4N} \dell x f(x)\phi(x) - s \del x g(x)\phi(x) 
  + 2Nu\delta(x  - \frac{1}{2N})
\]

We can write 

\begin{align}
  \dot \mu_k 
  &= \frac{1}{4N}\int_{0^+}^{1^-}dx x^k \dell x [f(x)\phi(x,t)] & & \bigg\} I_1
  \nonumber \\
  \quad &- s\int_{0^+}^{1^-}dx x^k \del x [g(x)\phi(x,t)] & & \bigg\} I_2
  \nonumber \\
  \quad &+\frac{u}{(2N)^{k -1}} + \dot K_0 \delta_{0,k} + \dot K_1 
  \label{eq:dmk}
\end{align}

For $k = 0$, it follows 
\begin{align}
  \dot \mu_0 &= \frac{1}{4N} \dev x [f(x)\phi(x)]_{0^+}^{1^-}
  - s [g(x)\phi(x)]_{0^+}^{1^-}
  +2Nu + \dot K_0  + \dot K_1 
  \nonumber \\
 &= - \frac{\phi(0^+) + \phi(1^-)}{4N}
  +2Nu + \dot K_0  + \dot K_1
\end{align}

Because the number of sites is constant $\dot \mu_0 = 0$ then we can require

for $k = 0$ it follows 
\begin{align}
  \dot K_0 &= \frac{\phi(0^+)}{4N} - 2Nu \\
  \dot K_1 &= \frac{\phi(1^-)}{4N}
\end{align}

For $k \geq 1$ we can solve by parts the integrals \eqref{eq:dmk} as follows
\begin{align}
  I_2 &= -s\{[g(x)\phi(x)x^k]_{0^+}^{1^-} - 
  k \int_{0^+}^{1^-} dx x^{k -1} [g(x)\phi(x,t)]\}  
  \nonumber \\
  &= - sk \int_{0^+}^{1^-} dx x^{k -1} (h + (1 - 2 h)x) x (1 - x)\phi(x,t)  
  \nonumber \\
  &= sk \int_{0^+}^{1^-} dx [h(x^{k + 1} - x^{k}) + 
  (1 - 2 h)(x^{k + 2} - x^{k + 1})]\phi(x,t)
  \nonumber \\
  &= sk [h(\mu_{k + 1} - \mu_{k}) + 
  (1 - 2 h)(\mu_{k + 2} - \mu_{k + 1})]
  \label{eq:I2}
  \\
  I_1 &= \frac{1}{4N}\{[\dev x (f(x)\phi(x)x^k]_{0^+}^{1^-})] - 
      k \int_{0^+}^{1^-} dx x^{k -1} \dev x [f(x)\phi(x,t)]\}
  \nonumber \\
   &= \frac{1}{4N}\{[\dev x (f(x)\phi(x))x^k]_{0^+}^{1^-})] - 
      k[f(x)\phi(x)x^{k-1}]_{0^+}^{1^-})]
      \nonumber \\
  \quad &+ k(k -1) \int_{0^+}^{1^-} dx x^{k - 2} f(x)\phi(x,t)\}
  \nonumber \\
   &= \frac{1}{4N}\{\phi(1^-) + k(k -1) \int_{0^+}^{1^-} dx (x^{k - 1} - x^k ) \phi(x,t)\}
  \nonumber \\
   &= \dot K_1 - \frac{k(k-1)}{4N}(\mu_k - \mu_{k-1})
\end{align}

It follows 
\begin{align}
	\label{eq:dmu_k}
	\dot \mu_k = \frac{k(k-1)}{8N} \pi_{k-1} + \frac{sk}{4}\Gamma_{k,h} + \frac{1}{(2N)^{k-1}}u,
\end{align}
where $\pi_k = 2(\mu_k - \mu_{k+1})$ and $\Gamma_{k,h} = 2[h\pi_k + (1 - 2h)\pi_{k+1}]$.

The expected change in genetic load can be written as three components representing
the instantaneous contributions of selection, mutation and drift: 
\[
	\dot W = \dot W_s + \dot W_u + \dot W_N,
\]
where 
\begin{align}
  \label{eq:Ws}
	\dot W_s &= s \left[ \frac{s}{1}(h\Gamma_{1,h} + )1 - 2h)\Gamma_{2,h})\right], \\
	\dot W_u &= s(2hu), \nonumber\\
	\dot W_N &= s \left[ \frac{1 - 2h}{4N}\pi_1\right].\nonumber
\end{align}

When two populations diverge over time the mutation component, $\dot W_u$, does
not contribute to load differences between the populations since it is assumed
to be constant over time and shared among populations. The drift term, $\dot
W_N$, is explicitly dependent of population size and leads to increased
differentiation  between populations over time.  Nevertheless, in the case of
variants of additive effects, the drift component is equal to zero. More
importantly, the selection component, $\dot W_s$, defines the efficacy of
selection and is the main contribution to differences in load between large
populations over  time.  It is important to emphasize that  equation
\eqref{eq:Ws} can be easily estimated from real data since it only depends on
estimation of allele frequency moments.

The function $\dot W_s$ is called the "FIT" efficacy of selection given its
contribution to  the fitness increase theorem \citep{Gravel2016,Ewens2004}.
Under an additive  scenario  $\dot W_s$ is called the Morton efficacy
of selection and is equivalent to the measurement of the rate of change in the
average number of deleterious variants in a population.  

Equation \eqref{eq:Ws} can be used as a summary statistic to measure
differences of efficacy between two populations.  Nevertheless, when analyzing
admixture events, this formula is not informative about the dependence of the
efficacy of selection in the admixed population with the parameters of the
source population. We therefore propose to develop the formulation of the
efficacy of selection for admixed population using a similar approach as the
one proposed by \cite{Jouganous2017}, numerically evaluating the time dynamic
of the function $\phi(x,t)$ for multiple populations under complex demographic
scenarios including admixture events.  

\section{Efficacy of Selection of Admixed Population}%
\label{sec:ef_admix} 

In a single pulse admxiture model we can write the 
\[
  \phi(x_1, x_2, x_3) = 
  \phi(x_1, x_2) \delta(x_3 - \alpha_1 x_1 - \alpha_2 x_2) 
\] 

In the infinite allele model there is pontentially a infinity number of alleles
ate frequency 0 and 1 so it is convinient to write 
\[
\mu_{3,k} \equiv \int_{0}^{1}dx_3 x_3^k \phi(x_3,t)dx 
= \int_{0^+}^{1^-}dx_3 x_3^k \phi(x_3,t) + K_{3,0} \delta_{0,k} + K_{3,1}
\]
where $K_0$ and $K_1$ are the number of sites at frequency 0 and 1 and 
$\delta_{0,k}$ is the Kronecker's delta. 

\begin{align}
  \dot \mu_k 
  &= \int_{0^+}^{1^-}dx x^k \del t \phi(x,t) 
  + \dot K_0 \delta_{0,k} + \dot K_1 
  \nonumber \\
  &= \del t \int_{0^+}^{1^-}dx x^k \phi(x,t) 
  + \dot K_0 \delta_{0,k} + \dot K_1 
\end{align}

then we can calculate the moments of the admix population as
\begin{align}
  \mu_{3,k} &= \int_0^1 dx_1 dx_2 dx_3 x_3^k \phi(x_1, x_2) 
  \delta(x_3 - \alpha_1 x_1 - \alpha_2 x_2)
  \nonumber \\
  &= \int_0^1 dx_1 dx_2 (\alpha_1 x_1 +  \alpha_2 x_2) ^k \phi(x_1, x_2)
  \nonumber \\
  &= \int_0^1 dx_1 dx_2 \sum_{j=1}^k\binom{j}{k}
  (\alpha_1 x_1)^j(\alpha_2x_2)^{k -j} \phi(x_1, x_2)
\end{align}

The diffusion equation for 2 populations is given by 

\begin{align}
\label{eq:difeq_2p}
\frac{\partial}{\partial t} \phi(x_1, x_2, t) & \approx 
\frac{1}{4N_1}\frac{\partial^2}{\partial x_1^2} 
x_1 (1 - x_1) \phi(x_1, x_2,t) 
+ \frac{1}{4N_2}\frac{\partial^2}{\partial x_2^2} 
x_2 (1 - x_2) \phi(x_1, x_2,t) \\ \nonumber
& \quad -s\frac{\partial}{\partial x_1} 
(h + (1 - 2 h)x_1)
x_1 (1 - x_1) \phi(x_1,x_2,t) \\ \nonumber
& \quad -s\frac{\partial}{\partial x_2} 
(h + (1 - 2 h)x_2)
x_2 (1 - x_2) \phi(x_1,x_2,t) \\ \nonumber
& \quad + 2 N_1 u \delta(x_1 - \frac{1}{2N_1})
+ 2 N_2 u \delta(x_2 - \frac{1}{2N_2})
\end{align}

Writing the diffusion equation as 
\begin{align}
  \del t \phi (x_1,x_2) =  
  \frac{1}{4N_1} \dell x_1 f_1(x_1)\phi(x_1, x_2) - s \del x_1 g_1(x_1)\phi(x_1) 
  \nonumber \\
  \frac{1}{4N_2} \dell x_2 f_2(x)\phi(x_1, x_2) - s \del x_2 g_2(x_2)\phi(x) 
  \nonumber \\
  + 2N_1u\delta(x_1  - \frac{1}{2N_1})
  + 2N_2u\delta(x_2  - \frac{1}{2N_2})
\end{align}

We can write 

\begin{align}
  \dot \mu_{kj}^{12}
  &= \frac{1}{4N_1}\int_{0^+}^{1^-}dx_1 x_1^k \dell {x_1} [f_1(x_1)\psi_1^j(x_1,t)] 
  & & \bigg\} I_{1,1}
  \nonumber \\
  \quad&+ \frac{1}{4N_2}\int_{0^+}^{1^-}dx_2 x_2^j \dell {x_2} [f_2(x_2)\psi_2^k(x_2,t)] 
  & & \bigg\} I_{1,2}
  \nonumber \\
  \quad &- s\int_{0^+}^{1^-}dx_1 x_1^k \del x_1 [g_1(x_1)\psi_1^j(x_1,t)] 
  & & \bigg\} I_{2,1}
  \nonumber \\
  \quad &- s\int_{0^+}^{1^-}dx_2 x_2^j \del x_2 [g_2(x_2)\psi_2^k(x_2,t)] 
  & & \bigg\} I_{2,2}
  \nonumber \\
  \quad &+\frac{u}{(2N_1)^{k -1}} +\frac{u}{(2N_2)^{j -1}}
  + \dot K_0^{12} \delta_{0,k} + \dot K_1^{12} 
  \label{eq:dmk3}
\end{align}

where $\psi_1^j(x_1) = \int_{0^+}^{1^-}dx_2 x_2^j \phi(x_1,x_2)$.

For $k = 0$, it follows 
\begin{align}
  \dot \mu_{0,0}^{1,2} &= 
  \frac{1}{4N_1} \dev x_1 [f_1(x_1)\psi_1^0(x_1)]_{0^+}^{1^-}
  - s [g_1(x_1)\psi_1^0(x_1)]_{0^+}^{1^-}
  \nonumber \\
  &\quad + \frac{1}{4N_2} \dev x_2 [f_2(x_2)\psi_1^0(x_2)]_{0^+}^{1^-}
  - s [g_2(x_1)\psi_2^0(x_2)]_{0^+}^{1^-}
  \nonumber \\
  &\quad +2N_1u + 2N_2u + \dot K_{3,0}  + \dot K_{3,1}
  \nonumber \\
 &= - \frac{1}{4N_1}[\int_{0+}^{1-} dx_2\phi(0^+, x_2) + \int_{0+}^{1-}
   dx_2\phi(1^-,x_2)]
   \nonumber \\
  &\quad - \frac{1}{4N_2}[\int_{0+}^{1-} dx_1\phi(x_1, 0^+) + \int_{0+}^{1-}
   dx_1\phi(x_1, 1^-)]
  \nonumber \\
  &\quad +2N_1u + 2N_2u + \dot K_{3,0}  + \dot K_{3,1}
  \nonumber \\
\end{align}

Because the number of sites is constant $\dot \mu_0 = 0$ then we can require

for $k = 0$ it follows 
\begin{align}
  \dot K_{3,0} &= 
  \frac{1}{4N_1}\int_{0+}^{1-} dx_2\phi(0^+, x_2) + 
  \frac{1}{4N_2}\int_{0+}^{1-} dx_1\phi(x_1, 0^+) - 
  2u(N_1 + N_2) \\
  \dot K_{1,0} &= 
  \frac{1}{4N_1}\int_{0+}^{1-} dx_2\phi(1^-, x_2) + 
  \frac{1}{4N_2}\int_{0+}^{1-} dx_1\phi(x_1, 1^-) 
\end{align}


For $k \geq 1$ we can solve by parts the integrals \eqref{eq:dmk3} as follows
\begin{align}
  I_{2,1} &= -s\{[g_1(x_1)\psi_1^j(x_1)x_1^k]_{0^+}^{1^-} - 
  k \int_{0^+}^{1^-} dx_1 x_1^{k -1} [g_1(x_1)\psi_1^j(x_1,t)]\}  
  \nonumber \\
  &= - sk \int_{0^+}^{1^-} dx_1 dx_2 
  x_1^{k -1} x_2^j (h + (1 - 2 h)x_1) x_1 (1 - x_1)\phi(x_1, x_2, t)  
  \nonumber \\
  &= sk \int_{0^+}^{1^-} dx_1 dx_2 [h(x_1^{k + 1} - x_1^{k}) + 
  (1 - 2 h)(x_1^{k + 2} - x_1^{k + 1})]x_2^j\phi(x_1, x_2, t)
  \nonumber \\
  &= sk [h(\mu_{k + 1,j}^{12} - \mu_{k,j}^{12}) + 
  (1 - 2 h)(\mu_{k + 2,j}^{12} - \mu_{k + 1,j}^{12})]
  \label{eq:I21}
  \\
  I_{1,1} &= \frac{1}{4N_1}\{[\dev x_1 (f_1(x_1)\psi_1(x_1)x_1^k]_{0^+}^{1^-})] - 
      k \int_{0^+}^{1^-} dx_1 x_1^{k -1} \dev x_1 [f_1(x_1)\psi_1(x_1,t)]\}
  \nonumber \\
   &= \frac{1}{4N_1}\{[\dev x_1 (f_1(x_1)\psi(x_1))x_1^k]_{0^+}^{1^-})] - 
      k[f_1(x_1)\psi(x)x_1^{k-1}]_{0^+}^{1^-})]
      \nonumber \\
  \quad &+ k(k -1) \int_{0^+}^{1^-} dx_1 dx_2x^{k - 2} x_2^jf_1(x_1)\phi(x_1, x_2,t)\}
  \nonumber \\
   &= \frac{1}{4N_1}\{\int_{0^+}^{1^-} dx_2 x_2^j \phi(1^-,x_2) 
   + k(k -1) \int_{0^+}^{1^-} dx (x^{k - 1} - x^k )x_2^j \phi(x,t)\}
  \nonumber \\
   &= \frac{1}{4N_1}\int_{0^+}^{1^-} dx_2 x_2^j \phi(1^-,x_2) 
   - \frac{k(k-1)}{4N}(\mu_{k,j}^{12}- \mu_{k-1,j}^{12})
  %\nonumber \\
  % &= \dot K_1 - \frac{k(k-1)}{4N}(\mu_k - \mu_{k-1})
\end{align}

\footnotesize
\twocolumn
\begin{spacing}{1}
\bibliographystyle{abbrvnat85}
\bibliography{mend}
\end{spacing}

\end{document}



